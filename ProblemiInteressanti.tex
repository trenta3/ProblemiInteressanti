\documentclass[a4paper,NoNotes,GeneralMath]{stdmdoc}
% Data la struttura leggermente eccezionale di questo file rispetto agli altri si addice di più utilizzare una struttura basata su amsthm
\usepackage{amsthm}
\usepackage{hyperref}

\theoremstyle{definition}
\newtheorem{problema}{Problema}
\newtheorem{spunto}{Spunto}

\begin{document}
	\title{Spunti e Problemi Interessanti}
	\autodate

	\section {Introduzione}
	In questo file raccolgo le mie elucubrazioni varie e/o domande interessanti che mi pongo, ma alle quali non ho né il tempo né la capacità di rispondere. Man mano che ricevo soluzioni parziali o controesempi li aggiungo. Se qualcuno trova la risposta ad una delle seguenti cose me lo scriva! \\
	Inoltre in una cartella di questo repository trovate dei files interessanti (paper o dispense) che riguardano alcune cose di nicchia per un CdL in matematica, ma che ritengo comunque piuttosto sensato conoscere.

	\section {Problemi}
	\begin{problema}[Dimensione ed isomorfismo col duale]
		$V^{*} \simeq V$ ($V$ isomorfo al suo duale) $\implies V$ di dimensione finita?
		\begin{itemize}
			\item D.A. dice di averlo risolto. A quanto pare è vero...
		\end{itemize}
	\end{problema}
	
	\begin{problema}[Varietà e Compattificazione di Alexandroff]
		Supponiamo di avere una varietà topologica $X$ (gli analisti veri possono porsi l'analoga domanda leggendo differenziabile al posto di topologica, i geometri algebrici possono leggere varietà algebrica). Ne consideriamo la compattificazione di Alexandroff (quella che aggiunge un punto solo) $\hat{X}$. La domanda che ci poniamo è: $\hat{X}$ (che a priori è solo uno spazio topologico) ha ancora una struttura di varietà topologica / differenziale / algebrica?
	\end{problema}
	
	\begin{problema}[Varietà Differenziabili e Rivestimenti]
		Sia $M$ una varietà differenziabile, e $X$ uno spazio topologico. Supponiamo di avere $p: M \rar X$ un rivestimento. Ci chiediamo se è vero che allora $X$ è anche una varietà differenziabile.
		\begin{itemize}
			\item La domanda mi è stata originariamente posta da L.P., e la risposta dovrebbe grosso modo essere affermativa. Controllate che la vostra costruzione torni con l'esempio di $\bbR$ e di $S^1$
		\end{itemize}
	\end{problema}
	
	\begin{problema}[Campi tolopogici: compattificazione e proiettivizzato]
		Sia $K$ un campo topologico. Ci chiediamo se la sua compattificazione di Alexandroff ed il suo proiettivizzato ($K^2$ modulo la solita relazione di equivalenza), visti come spazi topologici, siano omeomorfi.
	\end{problema}
	
	\begin{problema}[Invarianti per similitudine con matrici di permutazione]
		Si considerino le matrici $n \times n$ a coefficienti in un campo $K$. Si chiede di determinare degli inviarianti (e possibilmente degli algoritmi per calcolarli) per la seguente relazione di equivalenza: $ M \simeq N \sse \exists P $ matrice di permutazione tale che $M = P N P^{-1}$. Per matrici di permutazione si intendono matrici ad entrate di soli uni e zeri e tali che su ogni colonna e su ogni riga vi sia esattamente un uno. \\
		Questo problema nasce dal considerare che molte strutture (ad esempio i grafi) hanno come "classe di automorfismi" il relabeling di alcune loro costituenti (ad esempio i nodi). \\
		Si può notare che gli invarianti classici per la similitudine devono essere rispettati (in particolare autovalori e molteplicità geometrico/algebriche, etc.) Si chiede di determinare un sistema completo di invarianti per questa relazione un po' più stretta. \\
		\begin{itemize}
			\item Si potrebbe approcciare il problema considerando l'azione di $S_n$ (gruppo simmetrico con $n$ elementi), visto che sono proprio le matrici di permutazione. Le matrici $n \times n$ ne sarebbero quindi una rappresentazione, si potrebbe provare a scomporla in irriducibili...
		\end{itemize}
	\end{problema}

	\section {Spunti}
	\begin{spunto}[Concetti metrici in ambito topologico ed ultracose]
		Come noto, concetti come la {\it completezza}, la {\it totale limitatezza} per uno spazio metrico e l'essere {\it uniformemente continua} per una funzione sono tutti concetti metrici, ovvero non è banale definirli solo in base alla topologia. \\
		Ispirato da \href{https://terrytao.wordpress.com/2012/04/02/a-cheap-version-of-nonstandard-analysis/}{questo post di Terry Tao} ed in particolare dall'Esercizio 3 ivi riportato, che dà una caratterizzazione della continuità uniforme in termini di oggetti nonstandard, mi chiedo se sia possibili dare simili definizioni anche per la totale limitatezza e la completezza, per poter adottare poi tali definizioni anche nel contesto di spazi topologici e vedere se ne viene fuori qualcosa di buono. \\
		In particolare, si vorrebbe prendere come spazio l'ultralimite dello spazio topologico, con al posto di $\bbN$ come indicato nel post di Tao, un insieme di indici abbastanza grande per poter tenere a bada anche spazi non N2.
	\end{spunto}
	
	\begin{spunto}[PID-dificato di un anello]
		In analogia con quanto accade, dato un dominio d'integrità $A$, per la costruzione del campo delle frazioni, mi chiedo se si possa anche costruire un {\it "PIDdificato"} di un dominio, ovvero un anello $B$, PID (Dominio ad ideali principali), tale che $A$ si immerga in $B$ e che funzioni più o meno come il campo delle frazioni.
		\begin{itemize}
			\item A.M. mi fa giustamente notare come la cosa più sensata da chiedere sia se esiste il funtore aggiunto al forgetful che va dalla categoria dei PID a quella dei domini, ovvero se per ogni dominio $A$ esiste $B$ PID tale che per ogni altro anello PID $C$ e per ogni morfismo di anelli da $A$ a $C$, tale funzione si possa estendere in modo unico ad una funzione da $B$ a $C$, che coincida con la precedente quando ristretta.
			\item G.B. ci fa sapere che tale cosa esiste già e si chiama normalizzazione. Non si può fare per tutti i domini, ma solo per quelli che sono integralmente chiusi nel campo delle frazioni.
		\end{itemize}
	\end{spunto}
	
	\begin{spunto}[Campi topologici e chiusura algebrica]
		Supponiamo di avere un campo $K$ che è anche uno spazio topologico. Allora ci chiediamo se si può dare una topologia sensata alla chiusura algebrica di $K$?
		\begin{itemize}
			\item A.M. e D.A. mi dicono che si può fare una cosa sensata, e che la topologia ha pure delle proprietà simpatiche rispetto a quella di $K$. [Me la devo ancora far dire per bene]
		\end{itemize}
	\end{spunto}
	
	\begin{spunto}["Analogo" del teorema di Rouché in Analisi Reale]
		La speranza sarebbe quella di avere una scatola nera (ruolo che nell'analisi complessa svolge il teorema di Rouché) per poter fare le varie tecnicate di analisi reale come il teorema della funzione inversa, il teorema della mappa aperta, e simili potendosi basare su questo lemma (ne metto una specie di enunciato, da specificare meglio se a qualcuno viene qualche idea più sensata): \\
		Sia $f: \Omega \rar \bbR$ una funzione continua ($\Omega \subseteq \bbR^n$ aperto) e sia $c \in \Omega$ uno zero di $f$ tale che $\de f\mid_c$ è invertibile. Allora $\exists \varepsilon > 0$ tale che $\forall h: \Omega \rar \bbR$ $\cC^\infty$ tali che in un intorno di $c$ abbiano distanza in norma infinito da $f$ minore o uguale ad $\varepsilon$ si possa trovare un intorno di $c$ tale che $f$ ed $f+h$ hanno lo stesso numero di zeri.
	\end{spunto}
	
	\begin{spunto}[Versione "pulita" e generale del Risultante]
		La "teoria del Risultante" viene solitamente fatta nel caso di {\it due} polinomi di {\it una} variabile sola, ed è una parte di tutta l'algebra commutativa che non si può saltare: infatti, pur essendo una barca di conti con il determinante di una matrice, una volta sviluppata diventa un potente strumento al quale si appoggiano molte dimostrazioni. Ovvero è un bel modo di fare le cose che viene usato come scatola nera, ma per dimostrarlo si utilizzano troppi conti. \\
		Quello che vorrei è vedere uno sviluppo pulito (ovvero senza conti osceni) e un pelo più generale del Risultante (So che esiste il multirisultante, ma i modi che ho visto di tirarlo fuori hanno ancora più conti e si basano sul risultante classico). In particolare ho trovato da qualche parte il seguente enunciato, che non era però dimostrato: \\
		Siano $F_1, \ldots, F_n \in K[x_1, \ldots, x_m]$ (dove $K$ è un campo) generici polinomi (sarà chiarito in seguito) di gradi rispettivamente $d_1, \ldots, d_n$. Allora {\bf esiste un unico polinomio} $R = R_{d_1, \ldots, d_n}^{(m)} \in \bbZ[\text{coeff}(F_1, \ldots, F_n)]$ tale che:
		\begin{enumerate}
			\item $\{F_1 = \ldots = F_n = 0\}$ ammette soluzione $\sse R\mid_{F_1, \ldots, F_n} = 0$ come polinomio
			\item $R\mid_{x_0^{d_0}, \ldots, x_n^{d_n}} = 1$
			\item $R$ è un polinomio irriducibile su $\bbC[x_1, \ldots, x_m]$
		\end{enumerate}
		dove con $R\mid_{\text{cose}}$ si intende il polinomio $R$ valutato nei coefficienti di ciò che compare sotto (infatti il polinomio $R$ è richiesto essere in $\bbZ$ dei coefficienti dei generici polinomi di grado fissato). Ciò che si intende con generici polinomi è che si possono trattare i loro coefficienti come "indeterminate" per ottenere quindi un polinomio nei coefficienti generici che si possa valutare per ogni specificazione di $F_1, \ldots, F_n$ \\
		L'enunciato è, inutile dirlo, in alcuni punto poco chiaro e/o ambiguo. Mi chiedo quindi se si riesca a dare una dimostrazione sensata di questo (o un enunciato leggermente modificato) e verificare se il polinomio che salta fuori abbia anche le altre proprietà del risultante classico.
%		\begin{itemize}
%			\item Una strategia poteva essere quella di partire dalla definizione di determinante di matrici rettangolari usando la caratterizzazione $f$ e $g$ hanno un fattore in comune $\sse \exists a, b \in K[x_1, \ldots, x_m]$ tali che $af + bg = 0$ e con $\Deg a < \Deg g$ e $\Deg b < \Deg f$. E sfruttando il fatto che questi $a$ e $b$ esistono se e solo se i vettori $x^\alpha f$ e $x^\beta g$ sono linearmente dipendenti (con $\alpha$ e $\beta$ si intendono multiindici con le dovute restrizioni di grado). \\
%			Questa strada spiacevolmente non si riesce ad aggiustare perché non esiste una funzione polinomiale nelle entrate di una matrice rettangolare che valga zero se e solo se la matrice non ha rango massimo. Infatti data la matrice $[a b]$ ciò vorrebbe dire di avere un polinomio in due variabili che si annulla soltanto nel punto $a = b = 0$ (che in campi algebricamente chiusi non può esistere)
%			Resta comunque aperto il problema per matrici più grosse (magari con quelle di dimensioni \ge delle 2x3 si riesce)
%		\end{itemize}
	\end{spunto}
	
	\begin{spunto}[Proprietà di finitezza (Compattezza, Connessione, Irriducibilità)]
		Quando si introducono i concetti di compattezza e connessione per spazi topologici e di irriducibilità per varietà algebriche si dimostrano svariati lemmi che sono concettualmente molto simili, e che riporto di seguito. Mi chiedo in che modo si possa dare una nozione generale (magari categorialmente) di queste proprietà di "finitezza", con la quale si possa magari dimostrare gli equivalenti dei seguenti lemmi: \\
		\begin{enumerate}
			\item Sia $f \in \cC^0$ e $X$ uno spazio topologico compatto. Allora $f(X)$ è compatto.
			\item Sia $f \in \cC^0$ e $X$ uno spazio topologico connesso. Allora $f(X)$ è connesso.
			\item Sia $f$ morfismo di varietà e $X$ una varietà irriducibile. Allora $f(X)$ è irriducibile (potrebbe non essere una varietà però la sua chiusura è irriducibile. \hrule
			\item Il prodotto di due spazi topologici compatti è compatto.
			\item Il prodotto di due spazi topologici connessi è conesso.
			\item Il prodotto di due varietà irriducibile è irriducibile. (Qui si intende prodotto categorico) \hrule
			\item Se un insieme è connesso anche la sua chiusura è connessa
			\item Se un insieme è irriducibile anche la sua chiusura è irriducibile (ed è una varietà algebrica)
			\item Ci chiediamo se sia vero anche per i compatti, ovvero se un insieme è compatto ma non chiuso, la sua chiusura è compatta? \\
				\begin{itemize}
					\item D.A. ha trovato un controesempio: Consideriamo l'intervallo $[0,1]$ con numerabili copie dell'elemento $\frac{1}{2}$, ovvero lo spazio insiemisticamente formato da questi elementi dove una base degli aperti sono gli intervalli aperti $(a, b)$ con $0 \le a < b \le 1$ dove al posto di $\frac{1}{2}$ si può prendere una qualsiasi copia di $\frac{1}{2}$. \\
					Ora, se consideriamo il sottospazio $[0,1]$ "standard" (con una sola copia di $\frac{1}{2}$) si ha che esso ha la stessa topologia di quella euclidea, ed è quindi compatto. La chiusura coincide con tutto il suddetto spazio topologico, che però non è compatto: infatti posso considerare il ricoprimento con gli aperti $[0, 1]$ dove ogni volta scelgo una sola copia di $\frac{1}{2}$ ogni volta diversa dalle altre. In questo modo ogni unione finita copre solo un numero finito di copie di $\frac{1}{2}$.
				\end{itemize}
				\hrule
			\item Sia $f: X \rar Y$ un'applicazione chiusa. Se $Y$ è compatto e $f^{-1}(y)$ è compatto $\forall y \in Y$ allora anche $X$ è compatto.
			\item Sia $Y$ uno spazio topologico connesso ed $f: X \rar Y$ una mappa $\cC^0$ e surgettiva tale che $f^{-1}(y)$ è connesso $\forall y \in Y$. Se $f$ è aperta oppure chiusa, allora anche $X$ è connesso.
			\item (Mi pare che sia vero ma al momento non trovo parti da cui stia scritto) Sia $Y$ un irriducibile e $f: X \rar Y$ morfismo dominante di varietà. Se $\forall y \in Y$ si ha $f^{-1}(y)$ è irriducibile allora anche $X$ è irriducibile.
		\end{enumerate}
		Inoltre mi è venuto in mente che la proprietà "Essere di Baer" potrebbe essere una di queste. $X$ si dice di Baer se, presa una quantità numerabile di chiusi $C_i$ a parte interna vuota, anche la loro unione ha parte interna vuota. Ci chiediamo allora se valgano le seguenti cose:
		\begin{enumerate}
			\item $f: X \rar Y$ continua e surgettiva. Se $X$ è di Baer allora anche $Y$ è di Baer
			\item Sia $X \subseteq Y$, e tale che $X$ è di Baer, allora $\overline{X}$ è di Baer
			\item Sia $f: X \rar Y$ surgettiva. Se $Y$ è di Baer e $\forall y \in Y \quad f^{-1}(y)$ è di Baer allora anche $X$ è di Baer.
		\end{enumerate}
	\end{spunto}
	
	\begin{spunto}[Completamenti di vario genere]
		Anche se non so troppo di preciso che cosa sia un ultraprodotto / ultralimite, dovrebbero essere vero che sia la compattificazione di Stone-Cech che il completamento metrico lo sono. Mi chiedevo di che altre proprietà godano queste cose e quali altri completamenti si possano fare con queste cose.
	\end{spunto}
	
	\begin{spunto}[Rappresentazione indotta e Transfer di Artin]
		Su internet ho trovato un'affermazione simile al fatto che il transfer di artin si potesse ottenere come determinante di una rappresentazione di $\bbF_2 [G]$. Cerco un modo di formalizzare questa cosa e di scriverla per bene. In particolare si viene portati a pensare questo guardando la definizione delle due cose e vedendo che gli striccheggi effettuati sono più o meno gli stessi. \\
		\begin{itemize}
			\item La seguente potrebbe essere una strada: si consideri $P$ sottogruppo di $G$. e si consideri l'azione di $G$ sulle classi laterali $G/P$. 
		\end{itemize}
	\end{spunto}
	
	%\begin{spunto}["Accoppiamenti di Dualità"]\end{spunto}, Quello nato dal caso delle rappresentazioni di gruppi e caratteri reali / caratteri razionali
	
	\section {Files Interessanti}
	Qui di seguito elenco e commento i files interessanti che trovate nella cartella opportuna:
	\begin{itemize}
		\item ({\it Combinatorial\_NullStellenSatz.pdf}) Come dice il nome, si tratta del NullStellenSatz combinatorio, un lemma un po' tecnico che dà però delle stime su un polinomio in più variabili in funzione del numero di zeri.
		\item ({\it descartes.pdf}) Un file sulla regola dei segni di Cartesio, con molti enunciati particolari. Ad esempio, se ho un polinomio $p(x) \in \bbR[x]$ e so che ha tutte le radici reali, allora la regola di cartesio mi permette in questo caso di determinare esattamente quante sono negative e quante positive. Questo ad esempio consente di determinare la segnatura di una matrice simmetrica a coefficienti reali, dopo averne calcolato il polinomio caratteristico.
		\item ({\it Prime\_Ideal\_Principle.pdf}) Un paper che sviscera completamente l'argomento di quando un ideale massimale in una certa sottofamiglia di ideali sia un ideale primo dell'anello. Argomento che, anche se potrebbe non sembrare, ricorre abbastanza in algebra.
		\item ({\it Union\_of\_Vector\_Subspaces.pdf}) Risponde alla domanda: Quanti sottospazi vettoriali propri bisogna unire per poter coprire tutto lo spazio?
	\end{itemize}
	
	\section {Trucchi Vari}
	Riporto qualche trucchetto che potrebbe tornare utile in svariati casi
	\begin{itemize}
		\item Riguarda soprattutto l'Algebra Lineare. A volte ci si può aggiungere qualche ipotesi di comodo di tipo polinomiale sfruttando il fatto che $K[x_1, \ldots, x_n]$ è un dominio d'integrità. Ne do' un paio di esempi: prendiamo lo spazio delle matrici $\cM_n(K)$ $n \times n$ a coefficienti in $K$ e consideriamo le entrate della matrice come le incognite $a_{11}, a_{12}, \ldots, a_{1n}, a_{21}, \ldots, a_{nn}$. \\
		Vogliamo ad esempio mostrare che il polinomio caratteristico di $AB$ è uguale a quello di $BA$ (si, incredibile). Siamo in grado di dimostrarlo agevolmente se almeno una delle due matrici è invertibile (WLOG $B$). Infatti si ha $\Det(AB - t \Id) = \Det(A - t B^{-1}) \Det(B) = \Det(B) \Det(A - tB^{-1}) = \Det(BA - t)$ (usando la moltiplicatività del determinante). Vorremmo però mostrarlo per tutte le matrici. \\
		Notiamo allora che, assegnate le incognite $a_{ij}$ e $b_{ij}$ alle entrate delle due matrici $A$ e $B$, le entrate della matrice prodotto $AB$ sono polinomiali nelle entrate di $A$ e di $B$ (fare il conto). Inoltre si ha che anche il polinomio caratteristico di una matrice dipende in maniera polinomiale dalle entrate, visto che il determinante lo fa (convincersene scrivendo la formula del determinante con le permutazioni). Ci stiamo allora chiedendo se il polinomio $p(a_{ij}, b_{ij}, t) = \Det(AB - t \Id) - \Det(BA - t\Id)$ sia nullo oppure no. \\
		Sicuramente $p(a_{ij}, b_{ij}, t) \cdot \Det(B)$ è un polinomio sempre nullo (se $B$ è invertibile è nullo il primo fattore, altrimenti si annulla il secondo) e quindi per il principio di identità dei polinomi in più variabili (se $K$ è infinito, ma vale anche senza, aspettate un attimo) è il polinomio con tutti i coefficienti nulli. Ma sappiamo inoltre che il secondo fattore non può essere il polinomio nullo, visto che se metto $B = \Id$ (identità) allora $\Det(B) = 1 \neq 0$, quindi (visto che è un dominio d'integrità) si ha $p(a_{ij}, b_{ij}, t) \equiv 0$ e quindi abbiamo dimostrato la tesi. \\
		Se $K$ non fosse infinito, notiamo che possiamo passare alla chiusura algebrica tenendoci le stesse matrici ed il polinomio $p$ rimane invariato (come polinomio), ed ora possiamo applicare il principio di identità per ottenere che $p$ è nullo. \\
		Questo trucco, unito alla potenza del Risultante di polinomiarizzare alcuni enunciati, è piuttosto potente. Ad esempio, ci possiamo prendere l'ipotesi di comodo che una matrice abbia tutti gli autovalori distinti (in una chiusura algebrica), visto che questa condizione si scrive come $\text{Ris}(f(x), f'(x)) = 0$ dove Ris è il risultante e $f$ è il polinomio caratteristico della matrice. Questa condizione è, ancora una volta, polinomiale.
		\item Per quanto riguarda il passare da una disuguaglianza "debole" ad una con costanti migliori o di tipo simile, consiglio la lettura dell'\href{https://terrytao.wordpress.com/2007/09/05/amplification-arbitrage-and-the-tensor-power-trick/}{articolo} sul blog di Terry Tao, in cui fa un excursus su vari metodi, tutti piuttosto interessanti.
	\end{itemize}
	
	\section {Teoremi Carini}
	Riporto di seguito alcuni teoremi "esotici" che mi piacciono particolarmente
	\begin{itemize}
		\item ({\bf Lucas-Gauss}) Dice che dato un polinomio $p(x) \in \bbC[x]$ si ha che gli zeri della sua derivata $p'(x)$ sono contenuti nell'inviluppo convesso degli zeri di $p(x)$. Una cosa che mi sorprende abbastanza.
	\end{itemize}
\end{document}
