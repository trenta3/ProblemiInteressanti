\documentclass[a4paper,NoNotes,GeneralMath]{stdmdoc}

\begin{document}
	\section*{Matrici e Polinomi Minimi}
	\Altro{Cosa vogliamo mostrare?} Mostreremo che, dato un campo $\bbK$ infinto ed un polinomio $q(t) \in \bbK[t]$, esiste una matrice $A \in \kM(r, \bbK)$ tale che il suo polinomio minimo $m_A (t) = q(t)$. \\
	\begin{itemize}
	\item[CASO 1] Supponiamo che $q(t)$ sia irriducibile in $\bbK[t]$, e sia $\bbK^{*}$ il campo di spezzamento di $q$ su $\bbK$. Allora si ha che $$\bbK^{*} \cong \frac{\bbK[t]}{q(t)} \cong \bbK^{n+1} $$ dove $n = \deg q$. Vogliamo mostrare che esiste un morfismo $\varphi$ di campi iniettivo tra $\bbK^{*}$ e $\kM(n+1, \bbK)$. Siano $f$ l'isomorfismo di campi tra $\bbK^{*}$ e $\frac{\bbK[t]}{q(t)}$ e $g$ l'isomorfismo di spazi vettoriali tra $\frac{\bbK[t]}{q(t)}$ e $\bbK^{n+1}$. Definiamo $$\varphi(y) = \left( g(f(y)) \middle| g(xf(y)) \middle| \cdots \middle| g(x^n f(y)) \right)$$ \\ Si verifica piuttosto agevolmente che $\varphi$ è un morfismo di campi. L'iniettività segue dal fatto che $\varphi(1) = I$, ovvero che non tutto viene mandato in $0$. \\ Ora, siccome $\varphi$ è iniettivo, conserva i polinomi minimi. Sia $\alpha \in \bbK^{*}$ e $m_\alpha \in \bbK^{*}[t]$ il polinomio minimo di $\alpha$, $m_{\varphi(\alpha)}$ quello di $\varphi(\alpha)$. Allora $0 = \varphi(m_\alpha(\alpha)) = m_\alpha(\varphi(\alpha)) \implies m_{\varphi(\alpha)} \mid m_\alpha$, analogamente per l'altra divisione. Quindi $m_\alpha = m_{\varphi(\alpha)}$. \\ Quindi, se $\alpha$ è radice di $q$, siccome $q$ è irriducibile, si ha che anche il polinomio minimo di $\varphi(\alpha)$ è $q$.
	\item[CASO 2] $q(t) = r(t)^s$, con $r$ irriducibile. Dimostriamo per induzione che esiste la matrice che vogliamo. Per $s=1$ siamo nel caso precedente. \\ Per passare da $s$ a $s+1$, sia $M_s$ la matrice corrispondente al passo $s$ e consideriamo la matrice $M_{s+1} = \left( \begin{array}{c|c} M_s & I \\ \hline 0 & M_s \end{array} \right)$. Si verifica facilmente che $p(M_{s+1}) = \left( \begin{array}{c|c} p(M_s) & p'(M_s) \\ \hline 0 & p(M_s) \end{array} \right)$. \\ Calcolando ora $r(M_{s+1})^s+1 = 0$, quindi $m_{M_{s+1}} \mid r(t)^{s+1}$. D'altro canto $p(M_{s+1}) = 0 \implies p(M_{s}) = 0, p'(M_s) = 0$, quindi per ipotesi induttiva $r(t)^s \mid p, r(t)^s \mid p' \implies r(t)^{s+1} \mid p$. Perciò $r(t)^{s+1}$ è proprio il polinomio minimo di $M_{s+1}$.
	\item[CASO 3] $q(t) = \prod_{i=1}^k p_i(t)^{\beta_i}$. Riduciamoci al caso $p_i(t)$ irriducibile $\forall i$. Allora la matrice che funziona è $M = \left( \begin{array}{c|c|c} M_1 & & \\ \hline & \ddots & \\ \hline & & M_k \end{array} \right)$, poiché $m_M = \text{mcm }(m_{M_1}, \ldots, m_{M_k})$.
	\end{itemize}
\end{document}
